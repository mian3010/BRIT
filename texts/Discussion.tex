\section{Discussion}
The power all this gathering of information gives companies can be quite scary. The smartphones everyone has in their pockets posses enormous power to surveil us. And the technological development has not stopped there. New advancements give companies even more power over us. This part of the paper will try to discuss how this information gives companies tremendous power, and how companies might use the information. It will try to cite sources stating where this information has already been used to support the discussion.

\subsection{Google}
Google has worked hard to earn our trust, and they need it. As they expand their network of services (mostly free ones) a lot of people have voiced concerns about their sovereignty. They can, if they wish control information flow for the entire world.

Google has, for a long time now, bought different companies and converted the companies products into products of their own. Since February 2001 Google has acquired 151 companies. Lately though, the rate of companies bought have increased substantially and all the acquired companies have had roughly the same area of expertise: Artificial Intelligence (AI)\cite{website:google-buys}

Google has been forced by one of the companies to create an ethics board dealing with the ethics of the technologies they have.\cite{website:google-ethics} Ethics is very important when a company has so much power as Google has. The main question here is if we can trust Google to not misuse the power they have. Almost everything they do grant them some sort of power, and while most of the powers they have may seem harmless when you combine them you see just how immensely powerful Google is. Can we trust them not to abuse their power?

Many might answer no, but so far Google has not done anything significantly wrong with their power. They might be the best company in the bunch to have such power. I still find it very disturbing to think about.

\subsection{Facebook}
Like Google, Facebook is another huge company with enough users and with the right usage to be able to gather a lot of information about us all. Everybody that uses Facebook might be subject to behavioral analysis. Facebook have even the ability to track which other websites you visit using their like/share buttons. You do not even have to use them. Just having them there grants Facebook the power to monitor you.\cite{website:facebook-tracking}

Similar questions arise from the situation with facebook. Can we trust them to not misuse their enormous power? Facebook has almost as much power as Google, they are just not that fond of talking about it. This is also an alarming thing.

\subsection{Government regulations}
One thing that neither Facebook nor Google can control is the government regulations they have to follow. Many governments are interested in the data that both Google and Facebook have. And they continuously try to get the data from them. Many laws exist that force both Google and Facebook to share some of their data with the respective governments.

Recently though, both Facebook and Google have released information stating the number of requests for data they have received from different countries, and how many they have complied to.\cite{website:facebook-requests}\cite{website:google-requests}

Some questions remain. Who can we trust, our governments or the companies? Which one do we wish to have the ultimate power? And should we allow the two biggest players to gain more power, eventually making them too big to knock down if they should start misusing their power?
