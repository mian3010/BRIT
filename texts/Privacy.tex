\section{Privacy}
Privacy is a term which might not be clearly defined, but something we all know the meaning of. It is what is at stake in the society at the moment. Many have tried to define privacy and the importance of it in human development.

The main focus of this paper is on privacy, and the consequences of the lack of privacy, and it is therefore relevant to define what privacy is. In doing this the paper references several sources to try to form a usable definition.

Edward Snowden is a former contractor for the NSA through Dell and later Booz Allen Hamilton who has leaked many confidential documents because it revealed that the American government has been spying on its own citizens. He has been prosecutedby the US government, and has fled to another country. He has no real education in the research of privacy, but his fixed opinions about privacy of information has become widely recognized by many.\cite{website:edward-snowden-basic}. He has stated the following in a public message for the United States:

\blockquote{``A child born today will grow up with no conception of privacy at all. They’ll never know what it means to have a private moment to themselves — an unrecorded, unanalyzed thought.\\ And that’s a problem, because privacy matters; privacy is what allows us to determine who we are and who we want to be.''\cite{interview:edward-snowden-christmas}}
He defines privacy as an unrecorded, unalayzed thought, and that the absense of it is a big problem for us.

Kirstine Kollerup Madsen, a journalist of the Danish newspaper Politiken supports his idea of privacy by refering to the definition of democracy:

\blockquote{``The basic pillar of democracy is the inviolable integrity of the individual. Human integrity extends beyond the physical body. In their thoughts and in their personal environments and communications, all humans have the right to remain unobserved and unmolested.''\cite{website:a-stand-for-democracy}}

The importance of privacy is well known. Many have tried to emphasize this importance to form governments and laws in its interest.
In 1890 an attorney and a lawyer wrote an article called  \lq The Right to Privacy\rq which turned out to be one of the most influential essays in the history of American law.\cite{website:the-right-to-privacy-info}
They explain how the the principle of the right to protection of ones person and property is very old, but that it sometimes need to be redefined. In early times this law only included the physical protection of one self and ones property, but at the time of the essay, the authors deemed it nessecary to extend it to include a mans spiritual nature, his feelings and his intellect. They interpret the right to life as the right to enjoy life. This means the right for protection of ones thoughts and ideas.In 1948 the commission of human rights wrote the \lq Universal Declaration of Human Rights\rq which also describes the basic human right to have privacy. It contained the following statement:

In 1948 the commission of human rights wrote the \lq Universal Declaration of Human Rights\rq which also describes the basic human right to have privacy. It contained the following statement:

\blockquote{No one shall be subjected to arbitrary interference with his privacy, family, home or correspondence, nor to attacks upon his honour and reputation. Everyone has the right to the protection of the law against such interference or attacks.\cite{website:un-human-rights}}

Member states of the United Nations should live by these regulations.


In this section we have defined privacy as being an unrecorded, unalyzed thought. Privacy is also a part of the integrity of a person. The integrity does not only regard the physical body of a person, but his mind as well.
Privacy is a norm that we all regard as important to our daily life and evolvement. Many parts of the world see privacy as a human right, that should be honored.

\newpage
