\section{Privacy}
Privacy is a term which might not be clearly defined, but something we all know the meaning of. It is what is at stake in the society at the moment. Many have tried to define privacy and the importance of it in human development.

The main focus of this paper is on privacy, and the consequences of the lack of privacy, and it is therefore relevant to define what privacy is. In doing this the paper references several sources to try to form a usable definition.

Edward Snowden is a former contractor for the NSA through Dell and later Booz Allen Hamilton who has leaked many confidential documents because it revealed that the American government has been spying on its own citizens. He has been prosecutedby the US government, and has fled to another country. He has no real education in the research of privacy, but his fixed opinions about privacy of information has become widely recognized by many.\footnote{\cite{website:edward-snowden-basic}}. He has stated the following in a public message for the United States:

\blockquote{``A child born today will grow up with no conception of privacy at all. They’ll
never know what it means to have a private moment to themselves — an
unrecorded, unanalyzed thought.\\
And that’s a problem, because privacy matters; privacy is what allows us to
determine who we are and who we want to be.''\footnote{\cite{interview:edward-snowden-christmas}}}
He defines privacy as an unrecorded, unalayzed thought, and that the absense of it is a big problem for us.

Kirstine Kollerup Madsen, a journalist of the Danish newspaper Politiken supports his idea of privacy by refering to the definition of democracy:

\blockquote{``The basic pillar of democracy is the inviolable integrity of the individual. Human integrity extends beyond the physical body. In their thoughts and in their personal environments and communications, all humans have the right to remain unobserved and unmolested.''\footnote{\cite{website:a-stand-for-democracy}}}



\newpage
