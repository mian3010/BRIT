\section{Actor-network theory}
ANT (Actor-network theory) is a common anthropological theory which has been explained by many well-known anthropologists.

This paper will focus on Bruno Latours, a french sociologist of science and anthropologists explainaition of actor-network theory.

According to actor-network theory (explained by Bruno Latour) everything is of the same size. No place is considered bigger than another, and no actor in the network bigger than the other. Everything is about the actors and the network between them. When an actor communicates with another actor he created a line between them. This actor might communicate with other actors forming a network of lines.
Bruno Latour uses the example of Microsoft. Bill Gates, the head of Microsoft is no bigger than any of his employees, it is his network that sets him apart. He is just another actor.\cite{bookref:actor-network}

Bruno Latour has further used actor-network theory to create a theory of his own. This is an opposition to the panopticon theory, the oligopticon. He believed that the panopticon could only work in a perfect environment, a utopia. The theory of the oligopticon is more of a star-like structure, but should not be seen as a way of building a prison, more of a way the society can function.

Where the panopticon focuses on an all-seing part for monitoring, the oligopticon focuses more on parts that see only a small bit of the whole picture. But what they see they see very well. Because of this the oligopticon can easily be blinded. If what the oligopticon sees are hidden the oligopticon is blinded entirely, whereas the panopticon would still see a lot of other things. With actor-network theory in mind the oligopticon can prove very powerful but overwhelming in information. When gathering the observations from many actors of the oligopticon you can end up seeing all, but with high detail. This is both good and bad.

\newpage
