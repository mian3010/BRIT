\section{Surveillance and self-surveillance}
The Panopticon is a theory by Jeremy Bentham in the eighteen century about the setup of a high-performing prison. It was meant to be able to have thousands of prisoners with minimal staff. It was constructed in Cuba, and ended up having a lot of corruption and cruel acts towards prisoners. \cite{website:panopticon-idea}.

The idea of the panopticon is to have a central tower in the middle of the prison, and cells all the way around it. The central tower is for guards, and should be constructed in a way, so that the cells cannot see into it, but so that the guards can observe the prison cells. Prisoners in the panopticon is supposed to always conform to the rules and behave in the desired way as someone \textit{might} be watching them at any given time. The initial idea is that the best-behaving prisoner is a prisoner that is constantly watched. As this is not possible the next best solution is to make the prisoner believe that he is always being watched, or never have the prisoner see any signs that he is not being watched.\cite[P. 250]{bookref:organization-theory}

The french philosopher Michel Foucalt used the theory of the panopticon to describe that surveillance generates self-surveillance. This is acheived through two mechanism. The gaze which sets up the expectation of surveillance, and the interiozation leads to self-monitoring.

Self-monitoring is also used throughout organizations to keep its employees in line. It may be more or less subtle.\cite[P. 251]{bookref:organization-theory} Usually the gaze if the panopticon lies in a companies use of different management tools. These include but are not limited to:
\begin{itemize}
  \item{Employee interviews}
  \item{Psycological tests}
  \item{Performance appraisals}
  \item{Employee assesments}
\end{itemize}

When the employees of a company knows that the company uses some or all of the above, the employee can easily have the impression that they are under surveillance, and will execute self-surveillance to perform as expected of them.\cite[P. 251]{bookref:organization-theory}

Bruno Latour, a french sociologist of science and anthropologist was skeptical about the panopticon theory, and instead created his own version based on the actor-network theory.
