\section{Introduction}
Many companies today put a great deal of effort into collecting information about its users and their use of the companies services. This information is stored, and can easily be gathered to completely describe a persons interaction with many things. Today, data is usually collected realtime, and this enables the companies to accurately pinpoint your location and your doings at any given moment. This introduces a lot moral dilemmas. We rely on the companies to treat our data with care, and not to misuse it. Often though, governments around the world can submit requests to the companies that hold our data, to be allowed access, and because of different laws they are granted that right often. Can we trust our governments with this kind of information?

The technological advances within the last 5 years has made this collection of data possible.
This paper will start out by explaining some of the theories that can be applied to the situation, to reflect upon the new technologies effect on the society.
\newpage
