\section{Analysis}
A lot of our daily interaction with technology has to do with the exchange of information. Some of the regular daily tasks we perform are relatively new (introduced within the last 10 years). Some of the tasks of our daily life that this paper is focusing on are the following:
\begin{itemize}
  \item{Navigating to a location}
  \item{Vocal communication}
  \item{Textual communication}
  \item{Checking the social networks}
  \item{Otherwise tracking your behavior for own good}
\end{itemize}

All of the above tasks can be performed on a smartphone, but can also be performed on other technological devices.

We have come to trust the information our devices provide us, and for the companies to not misuse the information we send them. Often we are not even aware of how much the companies track about us. The following sections this paper will try to use the described theories in the previous sections to analyze upon the companies use of information and the effect on the society.

\subsection{Video surveillance}
Video surveillance is widely used in the modern society. Every shop has its own cameras, and some countries even have video surveillance of their streets. This is an old technology by now, but a technology nonetheless. Therefore we can analyze its impact on our behavior, and which theories that apply to it.

The security cameras that are placed everywhere are not meant to be seeing one thing well, they are placed to have the overview of an area. By that definition they fit into the panopticon theory. The panopticon also specifies that the surveillance must be frictionless, and without the observee knowing whether or not there is someone watching, but there might be. It is simply not possible to look at all the video received from surveillance cameras, so we all know that we are not being watched all the time, but there is a chance that someone is watching. This also fits with the theory of the panopticon. People will behave more as expected because of cameras.

Sometimes just the thought that there might be cameras can be enough to highten security. After an event where security was not strong enough in capital of Thailand, Bankok the government said they had placed 47.000 security cameras to monitor the capital. It later turned out that most of them were fake. It was just the housing.\cite{website:bangkok-security}

However, video surveillance is not limited to the obvious kind. Everyone today carries a camera with them, and you might be filmed without knowing it. Most of us think that we control whether or not the camera is active, but truth is that we are not in control. Our smartphones might be taking pictures or even filming everywhere we go. There have been created some malware that takes picture at all times. Those pictures are then uploaded to a central server, and pieced together to form a 3D map of your surroundings. This data could them be available to professional burglars to map the inside of your house.\cite{website:placeraider}

When smartphones or other devices communicate with a central server, or with eachother it fits into the theory of the oligopticon. Each phone (driven by a human or acting on its own) is an actor of the system and they create a network.

We have defined privacy as an unrecorded, unalalyzed thought, and a part of a persons integrity. If those malicious apps only capture your surroundings then our privacy will only be violated to a small extent. However, often the idea of the hidden surveillance is to record you, and what you are doing. Thoughts are not limited to something inside your head. Your actions begin from a thought. A recorded action is by this logic also a recorded thought, and privacy is violated by those malicious apps.

\subsection{Voice surveillance}
Similar to video surveillance voice surveillance exists as well. It is very hard to know if you are being recorded or not.

With the sudden popularity of smartphones everyone has a voice recorder in their pocket, and the recording of sound is not limited to the vision of the phone. A phone can easily record conversations from ones pocket.\cite{website:carrier-iq}

The recordings of a device like a smartphone is quite limited though. It cannot record conversation happening far away from it. Because of this the subtle recording of speech fits into the oligopticon theory. There are many actors that sends data to a central. They do not have the big picture, but a focused part of surroundings.

The recording of conversations is a clear interferrence with our privacy according to our definition.

\subsection{Location surveillance}
A lot of the devices we carry with us has several different ways of pinpointing your location, and tracking it. After the release of many documents from Edward Snowden Washington Post says that the documents contain information that NSA tracks the location of many of us.\cite{website:nsa-track}.

The theories that apply to this form of surveillance is the same as the for voice surveillance, and is a violation of privacy.

\subsection{Augmented reality}
If one were to think like the companies that are interested in surveillance, and one would consider the disadvantages of using a smartphone to gather such data, what would the solution be? Which device would be able to gather as much information as possible using both video, sound and location tracking?

Consider the new project called Glass from Google. This is essentially a smartphone attached as glasses on your head. You control them using voice, and it has most of the capabilities of a smartphone (maybe even more). This makes this device extremely powerful in collecting data about everyone (not just its users)\cite{website:google-glass}

The fact that Glass has been mentioned many times in the media, and people seem scared that everything might be recorded, most people is now aware that they might be when Glass is released. This fits into the theory of the panopticon, but only this part of Glass does. Glass is not all-seing. It sees a lot, but might also miss a lot. The power in surveillance by using Glass lies in the network of actors. This fits better into the theory of the oligopticon.\cite{website:glass-fears}

When you are using Glass everything might be recorded, and when something is recorded (a thought expressed in an action) our privacy is violated.
